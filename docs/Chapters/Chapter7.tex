

\chapter{Trabajo futuro} % Main chapter title

\label{Futuro} % Change X to a consecutive number; for referencing this chapter elsewhere, use \ref{ChapterX}

%----------------------------------------------------------------------------------------
%	SECTION 1
%----------------------------------------------------------------------------------------


Una vez que se han descrito los resultados más relevantes que se han obtenido durante la realización de este trabajo, se sugiere una serie de puntos de líneas de continuidad para investigaciones futuras:
\begin{itemize}
    \item El estudio de nuevas técnicas de ataques físicos (mundo real) a sistemas de visión artificial basados en redes neuronales artificiales, utilizando algoritmos genéticos para identificar la ubicación que maximiza el error del clasificador en la distribución de las perturbaciones. Podría recibir como parámetro la imagen víctima y la forma de las pegatinas a disponer (rectangular, redonda), cantidad y tamaño.  
    \item Creación de una nueva metodología en la identificación de dígitos de placas matrículas con redes neuronales artificiales de manera robusta, filtrando las perturbaciones provenientes de ataques. Este podría corregir formas no habituales en dígitos previamente conocidos en cuanto a las proporciones de cada uno de ellos, de modo de maximizar la eficacia de su resultado.
    \item Generación de un nuevo marco de trabajo evaluador de robustez de modelos de visión artificial basados en redes neuronales artificiales. Este podría estudiar las fronteras de clasificación generadas y además a través de ataques éticos establecer el nivel de seguridad que posee el modelo analizado, con el fin de proporcionar al arquitecto de la red un indicador sobre el nivel de robustez que posee el modelo. Este podría establecer una métrica para esta característica que permita comparaciones%.\parencite{r5}
    
\end{itemize}



